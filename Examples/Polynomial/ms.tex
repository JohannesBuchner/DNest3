\documentclass[letterpaper, 11pt]{article}
\usepackage{graphicx}
\usepackage{natbib}
\usepackage[left=3cm,top=3cm,right=3cm]{geometry}

\renewcommand{\topfraction}{0.85}
\renewcommand{\textfraction}{0.1}
\parindent=0cm

\title{Fitting a Polynomial in DNest}
\author{Brendon J. Brewer}

\begin{document}
\maketitle

\abstract{A little demonstration of how to implement a model in DNest.}

\section{The Problem}
The generative model for the data is that it was produced from an underlying
4th-order polynomial plus Gaussian noise:
\begin{eqnarray}
y_i = a_0 + a_1 x + a_2 x^2 + a_3 x^3 + a_4 x^4 + \epsilon_i
\end{eqnarray}
where
\begin{eqnarray}
\epsilon_i \sim \mathcal{N}(0, \sigma^2).
\end{eqnarray}

We will assume that the order of the polynomial ($n=4$) and the noise level
$\sigma=0.2$ are known.

The first thing to do is make a DNest3 model. It's probably easiest to copy
the SpikeSlab directory and its contents.

\end{document}

